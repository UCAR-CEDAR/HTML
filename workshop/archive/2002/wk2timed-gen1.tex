\parskip.10in
\parindent0in
\section{Monday 01:00-03:00 PM, 17 June 2002, CEDAR Workshop}
\index{Monday 01:00-03:00 PM, 17 June 2002, CEDAR Workshop}
\index{TIMED-CEDAR Workshop (Yee/Talaat)}

{\obeylines\obeyspaces
{\it Front Range Theatre
\vskip.10truein
\parskip0pt
Sam Yee (sam.yee@jhuapl.edu) John Hopkins University Applied Physics Lab
Elsayed Talaat (elsayed.talaat@jhuapl.edu) John Hopkins University Applied Physics Lab
\vskip.10truein
}}

The NASA 
Thermosphere-Ionosphere-Mesosphere Energetics and Dynamics 
(http://www.timed.jhuapl.edu) TIMED
satellite was successfully launched December 7, 2001.  The overall TIMED
mission also includes numerous NSF/CEDAR ground-based collaborative partners.  TIMED studies the temporal and spatial variations of the basic
atmospheric structure and energy balance between 60 and 180 kilometers.
This workshop will be dedicated to planning collaborations between
groundbased and satellite observational teams with an emphasis on validation
studies, as well as presentation of  preliminary scientific results.  There
will be short oral presentations on the status of the mission and
instruments.  

We would like to invite all interested to participate in the
discussions afterward.  For those of you who would like to make short
presentations on your planned collaborations and/or scienitific results
please contact 
(Sam.Yee@jhuapl.edu) Dr. Sam Yee
 (240-228-6206).

The TIMED instruments include: GUVI, a spatial scanning far ultraviolet
spectrograph that measures composition and temperature in the lower
thermosphere, as well as auroral energy inputs;  SABER, an infrared
radiometer, measures pressure, temperature, and infrared cooling rates in
the stratosphere, mesosphere, and lower thermosphere; SEE, a spectrometer
and a suite of photometers, measures incoming solar irradiance; and TIDI, a
Fabry-Perot interferometer, measures horizontal vector winds, temperature,
and composition in the mesosphere and lower thermosphere.  We also have
numerous groundbased NSF/CEDAR partners conducting joint 
(http://cedarweb.hao.ucar.edu/timed/timed.html) TIMED/CEDAR studies. 

For more information about the TIMED misssion, go to http://www.timed.jhuapl.edu.

